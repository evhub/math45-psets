\documentclass[12pt,letterpaper]{hmcpset}
\usepackage[margin=1in]{geometry}
\usepackage{graphicx}
\usepackage{amsmath,amssymb}
\usepackage{enumerate}

% info for header block in upper right hand corner
\name{}
\class{Math 45 - Section ---}
\assignment{Homework 3}
\duedate{Friday, April 8, 2016}

\newcommand{\RR}{\mathbb{R}}

\begin{document}

\problemlist{D.\{1,2,3,4,5,6,7,8\}}

\begin{problem}[D1]
    For each of the following differential
    equations find its general solution:    
    \begin{enumerate}[(a)]
        \item $y'' - y' - 12y = 0$
        \item $y'' + 3y' + y = 0$
        \item $y'' + 3y' + 3y = 0$
    \end{enumerate}
\end{problem}

\begin{solution}
\end{solution}
\clearpage

\begin{problem}[D2]
    For each of the following differential
    equations find its general solution:    
    \begin{enumerate}[(a)]
        \setcounter{enumi}{3}
        \item $y'' + 4y' + 4y = 0$
        \item $y'' + 2y' + 2y = 0$
        \item $4y'' + 12y' + 9y = 0$
    \end{enumerate}
\end{problem}

\begin{solution}
\end{solution}
\clearpage

\begin{problem}[D3]
    Find the solution that satisfies $y(0) = 1$, $y'(0)
    = 0$ in equations (a), (c) and (f) of Problems
    D1 and D2.
\end{problem}

\begin{solution}
    \vfill
\end{solution}
\clearpage

\begin{problem}[D4]
    Consider the initial value problem
    \[
        \begin{cases}
            2y'' - y' - 3y = 0\\
            y(0) = A\\
            y'(0) = B.
        \end{cases}
    \]
    \begin{enumerate}[(a)]
        \item  Find the solution to this problem
            in terms of A and B.
        \item For what values of A and B does the
            solution tend to 0 as $t\to\infty$?
        \item For what values of A and B does the
            solution tend to $\infty$ as $t\to\infty$?
    \end{enumerate}
\end{problem}

\begin{solution}
    \vfill
\end{solution}
\clearpage

\begin{problem}[D5]
    For each of the following initial value problems,
    determine the largest interval on which a unique
    twice differentiable solution, $y(t)$, is certain 
    to exist.
    \begin{enumerate}[(a)]
        \item $ty'' + 3y = t^3,\quad y(1)=1,\quad y'(1)=2$
        \item $(t-3)y'' + ty' + \ln|t|y = 0, \quad
            y(1)=0,\quad y'(1)=1$
    \end{enumerate}
\end{problem}

\begin{solution}
    \vfill
\end{solution}
\clearpage

\begin{problem}[D6]
    Consider the initial value problem
    \begin{align}
        \begin{cases}
            y''=-y\\
            y(0) = \cos(a)\\
            y'(0) = -\sin(a).
        \end{cases}\label{d6}
    \end{align}
    \begin{enumerate}[(a)]
        \item Find the solution to (\ref{d6}).
        \item Prove $\cos(t + a)$ also solves the
            above initial value problem.
        \item Use the uniqueness theorem to prove 
            that $\cos(t + a) = \cos(t) \cos(a) - 
            \sin(t) \sin(a)$ for any $t,a\in\RR$.
        \item Now show that another form of the general
            solution to (\ref{d6}) is $b\cos(t+a)$.
    \end{enumerate}
\end{problem}

\begin{solution}
    \vfill
\end{solution}
\clearpage

\begin{problem}[D7]
    Prove that if $a, b, c$ are three different 
    real numbers then $e^{ax},e^{bx},e^{cx}$ are
    linearly independent functions.
\end{problem}

\begin{solution}
    \vfill
\end{solution}
\clearpage

\begin{problem}[D8 (Conservation of Energy)]
     Let $V(x)$ be a twice continuously differentiable
     function. Using Newton’s second law of motion we
     can model a marble of mass 1 rolling along the graph
     of $V(x)$ starting at position $x_0$ and with initial
     velocity $v_0$ (where positive velocity indicates
     moving to the right) with the initial value problem
     \begin{align}
         \begin{cases}
            x'' = -V'(x)\\
            x(0)=x_0\\
            x'(0)=v_0.
         \end{cases}\label{d8}
     \end{align}
     This model assumes that the marble does not fly off 
     the surface of the graph and that there is no friction.
     The function $V(x)$ in this context is referred to as 
     a potential function. \textit{Note: A particularly 
     interesting potential is $V (x) = \frac{1}{2}kx^2$ with 
     $k > 0$, the potential energy for a spring. In this case the
     equation becomes $x'' = −kx$, whose fundamental solutions
     are periodic. This implies that a spring system is equivalent 
     to a marble rolling up and down a parabola.}
     \begin{enumerate}[(a)]
         \item Show that any solution, $x(t)$, to (\ref{d8})
             conserves the energy, i.e. $E$ is constant,
             \[
                 E(t) = \frac{1}{2}\left(x'(t)\right)^2 + V(x(t)).
             \]
         \item Show that if the right hand side in (\ref{d8}) is 
            replaced by $-V'(x)-cx'$, $c>0$ the energy is dissipated.
            \textit{Note: This DOES NOT model friction, as friction
            is not proportional to the velocity of the marble.
            Since the force of friction is discontinuous (static
            vs kinetic friction) it causes issues for existence 
            of solutions.}
     \end{enumerate}
\end{problem}

\begin{solution}
    \vfill
\end{solution}
\clearpage

\end{document}
